% Options for packages loaded elsewhere
\PassOptionsToPackage{unicode}{hyperref}
\PassOptionsToPackage{hyphens}{url}
%
\documentclass[
]{article}
\usepackage{lmodern}
\usepackage{amsmath}
\usepackage{ifxetex,ifluatex}
\ifnum 0\ifxetex 1\fi\ifluatex 1\fi=0 % if pdftex
  \usepackage[T1]{fontenc}
  \usepackage[utf8]{inputenc}
  \usepackage{textcomp} % provide euro and other symbols
  \usepackage{amssymb}
\else % if luatex or xetex
  \usepackage{unicode-math}
  \defaultfontfeatures{Scale=MatchLowercase}
  \defaultfontfeatures[\rmfamily]{Ligatures=TeX,Scale=1}
\fi
% Use upquote if available, for straight quotes in verbatim environments
\IfFileExists{upquote.sty}{\usepackage{upquote}}{}
\IfFileExists{microtype.sty}{% use microtype if available
  \usepackage[]{microtype}
  \UseMicrotypeSet[protrusion]{basicmath} % disable protrusion for tt fonts
}{}
\makeatletter
\@ifundefined{KOMAClassName}{% if non-KOMA class
  \IfFileExists{parskip.sty}{%
    \usepackage{parskip}
  }{% else
    \setlength{\parindent}{0pt}
    \setlength{\parskip}{6pt plus 2pt minus 1pt}}
}{% if KOMA class
  \KOMAoptions{parskip=half}}
\makeatother
\usepackage{xcolor}
\IfFileExists{xurl.sty}{\usepackage{xurl}}{} % add URL line breaks if available
\IfFileExists{bookmark.sty}{\usepackage{bookmark}}{\usepackage{hyperref}}
\hypersetup{
  pdftitle={Домашнее задание},
  hidelinks,
  pdfcreator={LaTeX via pandoc}}
\urlstyle{same} % disable monospaced font for URLs
\usepackage[margin=1in]{geometry}
\usepackage{color}
\usepackage{fancyvrb}
\newcommand{\VerbBar}{|}
\newcommand{\VERB}{\Verb[commandchars=\\\{\}]}
\DefineVerbatimEnvironment{Highlighting}{Verbatim}{commandchars=\\\{\}}
% Add ',fontsize=\small' for more characters per line
\usepackage{framed}
\definecolor{shadecolor}{RGB}{248,248,248}
\newenvironment{Shaded}{\begin{snugshade}}{\end{snugshade}}
\newcommand{\AlertTok}[1]{\textcolor[rgb]{0.94,0.16,0.16}{#1}}
\newcommand{\AnnotationTok}[1]{\textcolor[rgb]{0.56,0.35,0.01}{\textbf{\textit{#1}}}}
\newcommand{\AttributeTok}[1]{\textcolor[rgb]{0.77,0.63,0.00}{#1}}
\newcommand{\BaseNTok}[1]{\textcolor[rgb]{0.00,0.00,0.81}{#1}}
\newcommand{\BuiltInTok}[1]{#1}
\newcommand{\CharTok}[1]{\textcolor[rgb]{0.31,0.60,0.02}{#1}}
\newcommand{\CommentTok}[1]{\textcolor[rgb]{0.56,0.35,0.01}{\textit{#1}}}
\newcommand{\CommentVarTok}[1]{\textcolor[rgb]{0.56,0.35,0.01}{\textbf{\textit{#1}}}}
\newcommand{\ConstantTok}[1]{\textcolor[rgb]{0.00,0.00,0.00}{#1}}
\newcommand{\ControlFlowTok}[1]{\textcolor[rgb]{0.13,0.29,0.53}{\textbf{#1}}}
\newcommand{\DataTypeTok}[1]{\textcolor[rgb]{0.13,0.29,0.53}{#1}}
\newcommand{\DecValTok}[1]{\textcolor[rgb]{0.00,0.00,0.81}{#1}}
\newcommand{\DocumentationTok}[1]{\textcolor[rgb]{0.56,0.35,0.01}{\textbf{\textit{#1}}}}
\newcommand{\ErrorTok}[1]{\textcolor[rgb]{0.64,0.00,0.00}{\textbf{#1}}}
\newcommand{\ExtensionTok}[1]{#1}
\newcommand{\FloatTok}[1]{\textcolor[rgb]{0.00,0.00,0.81}{#1}}
\newcommand{\FunctionTok}[1]{\textcolor[rgb]{0.00,0.00,0.00}{#1}}
\newcommand{\ImportTok}[1]{#1}
\newcommand{\InformationTok}[1]{\textcolor[rgb]{0.56,0.35,0.01}{\textbf{\textit{#1}}}}
\newcommand{\KeywordTok}[1]{\textcolor[rgb]{0.13,0.29,0.53}{\textbf{#1}}}
\newcommand{\NormalTok}[1]{#1}
\newcommand{\OperatorTok}[1]{\textcolor[rgb]{0.81,0.36,0.00}{\textbf{#1}}}
\newcommand{\OtherTok}[1]{\textcolor[rgb]{0.56,0.35,0.01}{#1}}
\newcommand{\PreprocessorTok}[1]{\textcolor[rgb]{0.56,0.35,0.01}{\textit{#1}}}
\newcommand{\RegionMarkerTok}[1]{#1}
\newcommand{\SpecialCharTok}[1]{\textcolor[rgb]{0.00,0.00,0.00}{#1}}
\newcommand{\SpecialStringTok}[1]{\textcolor[rgb]{0.31,0.60,0.02}{#1}}
\newcommand{\StringTok}[1]{\textcolor[rgb]{0.31,0.60,0.02}{#1}}
\newcommand{\VariableTok}[1]{\textcolor[rgb]{0.00,0.00,0.00}{#1}}
\newcommand{\VerbatimStringTok}[1]{\textcolor[rgb]{0.31,0.60,0.02}{#1}}
\newcommand{\WarningTok}[1]{\textcolor[rgb]{0.56,0.35,0.01}{\textbf{\textit{#1}}}}
\usepackage{graphicx}
\makeatletter
\def\maxwidth{\ifdim\Gin@nat@width>\linewidth\linewidth\else\Gin@nat@width\fi}
\def\maxheight{\ifdim\Gin@nat@height>\textheight\textheight\else\Gin@nat@height\fi}
\makeatother
% Scale images if necessary, so that they will not overflow the page
% margins by default, and it is still possible to overwrite the defaults
% using explicit options in \includegraphics[width, height, ...]{}
\setkeys{Gin}{width=\maxwidth,height=\maxheight,keepaspectratio}
% Set default figure placement to htbp
\makeatletter
\def\fps@figure{htbp}
\makeatother
\setlength{\emergencystretch}{3em} % prevent overfull lines
\providecommand{\tightlist}{%
  \setlength{\itemsep}{0pt}\setlength{\parskip}{0pt}}
\setcounter{secnumdepth}{-\maxdimen} % remove section numbering
\usepackage[russian]{babel}
\usepackage{hyperref}
\hypersetup{ colorlinks = true, urlcolor = blue}
\ifluatex
  \usepackage{selnolig}  % disable illegal ligatures
\fi

\title{Домашнее задание}
\author{}
\date{\vspace{-2.5em}}

\begin{document}
\maketitle

\hypertarget{ux437ux430ux434ux430ux43dux438ux435-1}{%
\subsubsection{Задание 1}\label{ux437ux430ux434ux430ux43dux438ux435-1}}

\begin{enumerate}
\def\labelenumi{\arabic{enumi}.}
\tightlist
\item
  Загрузите файл \texttt{housing.csv} с данными по ценам на квартиры в
  пригородах Бостона (одна строка -- один пригород) и сохраните его в
  датафрейм. Описание показателей:
\end{enumerate}

\begin{Shaded}
\begin{Highlighting}[]
\NormalTok{dat }\OtherTok{\textless{}{-}} \FunctionTok{read.csv}\NormalTok{(}\StringTok{\textquotesingle{}D:/Skillbox/Analytics, Middle/housing.csv\textquotesingle{}}\NormalTok{, }\AttributeTok{sep=}\StringTok{\textquotesingle{}}\SpecialCharTok{\textbackslash{}t}\StringTok{\textquotesingle{}}\NormalTok{)}
\end{Highlighting}
\end{Shaded}

\begin{itemize}
\item
  \texttt{RM} -- среднее число комнат;
\item
  \texttt{LSTAT} -- процент жителей низкого материального статуса:
\item
  \texttt{MEDV} -- медианное значение цены на дом в тысячах долларов.

  Посмотрите на датафрейм -- откройте его в отдельном окне RStudio.
\end{itemize}

\begin{enumerate}
\def\labelenumi{\arabic{enumi}.}
\setcounter{enumi}{1}
\tightlist
\item
  Выведите на экран первые и последние строки в датафрейме. Выведите
  информацию о типах столбцов в таблице («структуру» датафрейма).
  Проверьте, что все столбцы с числовыми данными считались как числовые,
  а не текстовые.
\end{enumerate}

\begin{Shaded}
\begin{Highlighting}[]
\FunctionTok{head}\NormalTok{(dat)}
\end{Highlighting}
\end{Shaded}

\begin{verbatim}
##   OBS.       TOWN TOWN. TRACT      LON     LAT MEDV CMEDV    CRIM ZN INDUS CHAS
## 1    1     Nahant     0  2011 -70.9550 42.2550 24.0  24.0 0.00632 18  2.31    0
## 2    2 Swampscott     1  2021 -70.9500 42.2875 21.6  21.6 0.02731  0  7.07    0
## 3    3 Swampscott     1  2022 -70.9360 42.2830 34.7  34.7 0.02729  0  7.07    0
## 4    4 Marblehead     2  2031 -70.9280 42.2930 33.4  33.4 0.03237  0  2.18    0
## 5    5 Marblehead     2  2032 -70.9220 42.2980 36.2  36.2 0.06905  0  2.18    0
## 6    6 Marblehead     2  2033 -70.9165 42.3040 28.7  28.7 0.02985  0  2.18    0
##     NOX    RM  AGE    DIS RAD TAX PTRATIO      B LSTAT
## 1 0.538 6.575 65.2 4.0900   1 296    15.3 396.90  4.98
## 2 0.469 6.421 78.9 4.9671   2 242    17.8 396.90  9.14
## 3 0.469 7.185 61.1 4.9671   2 242    17.8 392.83  4.03
## 4 0.458 6.998 45.8 6.0622   3 222    18.7 394.63  2.94
## 5 0.458 7.147 54.2 6.0622   3 222    18.7 396.90  5.33
## 6 0.458 6.430 58.7 6.0622   3 222    18.7 394.12  5.21
\end{verbatim}

\begin{Shaded}
\begin{Highlighting}[]
\FunctionTok{tail}\NormalTok{(dat)}
\end{Highlighting}
\end{Shaded}

\begin{verbatim}
##     OBS.     TOWN TOWN. TRACT      LON     LAT MEDV CMEDV    CRIM ZN INDUS CHAS
## 501  501   Revere    90  1708 -70.9920 42.2380 16.8  16.8 0.22438  0  9.69    0
## 502  502 Winthrop    91  1801 -70.9860 42.2312 22.4  22.4 0.06263  0 11.93    0
## 503  503 Winthrop    91  1802 -70.9910 42.2275 20.6  20.6 0.04527  0 11.93    0
## 504  504 Winthrop    91  1803 -70.9948 42.2260 23.9  23.9 0.06076  0 11.93    0
## 505  505 Winthrop    91  1804 -70.9875 42.2240 22.0  22.0 0.10959  0 11.93    0
## 506  506 Winthrop    91  1805 -70.9825 42.2210 11.9  19.0 0.04741  0 11.93    0
##       NOX    RM  AGE    DIS RAD TAX PTRATIO      B LSTAT
## 501 0.585 6.027 79.7 2.4982   6 391    19.2 396.90 14.33
## 502 0.573 6.593 69.1 2.4786   1 273    21.0 391.99  9.67
## 503 0.573 6.120 76.7 2.2875   1 273    21.0 396.90  9.08
## 504 0.573 6.976 91.0 2.1675   1 273    21.0 396.90  5.64
## 505 0.573 6.794 89.3 2.3889   1 273    21.0 393.45  6.48
## 506 0.573 6.030 80.8 2.5050   1 273    21.0 396.90  7.88
\end{verbatim}

\begin{Shaded}
\begin{Highlighting}[]
\FunctionTok{str}\NormalTok{(dat)}
\end{Highlighting}
\end{Shaded}

\begin{verbatim}
## 'data.frame':    506 obs. of  21 variables:
##  $ OBS.   : int  1 2 3 4 5 6 7 8 9 10 ...
##  $ TOWN   : chr  "Nahant" "Swampscott" "Swampscott" "Marblehead" ...
##  $ TOWN.  : int  0 1 1 2 2 2 3 3 3 3 ...
##  $ TRACT  : int  2011 2021 2022 2031 2032 2033 2041 2042 2043 2044 ...
##  $ LON    : num  -71 -71 -70.9 -70.9 -70.9 ...
##  $ LAT    : num  42.3 42.3 42.3 42.3 42.3 ...
##  $ MEDV   : num  24 21.6 34.7 33.4 36.2 28.7 22.9 27.1 16.5 18.9 ...
##  $ CMEDV  : num  24 21.6 34.7 33.4 36.2 28.7 22.9 22.1 16.5 18.9 ...
##  $ CRIM   : num  0.00632 0.02731 0.02729 0.03237 0.06905 ...
##  $ ZN     : num  18 0 0 0 0 0 12.5 12.5 12.5 12.5 ...
##  $ INDUS  : num  2.31 7.07 7.07 2.18 2.18 2.18 7.87 7.87 7.87 7.87 ...
##  $ CHAS   : int  0 0 0 0 0 0 0 0 0 0 ...
##  $ NOX    : num  0.538 0.469 0.469 0.458 0.458 0.458 0.524 0.524 0.524 0.524 ...
##  $ RM     : num  6.58 6.42 7.18 7 7.15 ...
##  $ AGE    : num  65.2 78.9 61.1 45.8 54.2 58.7 66.6 96.1 100 85.9 ...
##  $ DIS    : num  4.09 4.97 4.97 6.06 6.06 ...
##  $ RAD    : int  1 2 2 3 3 3 5 5 5 5 ...
##  $ TAX    : int  296 242 242 222 222 222 311 311 311 311 ...
##  $ PTRATIO: num  15.3 17.8 17.8 18.7 18.7 18.7 15.2 15.2 15.2 15.2 ...
##  $ B      : num  397 397 393 395 397 ...
##  $ LSTAT  : num  4.98 9.14 4.03 2.94 5.33 ...
\end{verbatim}

\begin{enumerate}
\def\labelenumi{\arabic{enumi}.}
\setcounter{enumi}{2}
\tightlist
\item
  Выведите на экран сводные характеристики всех столбцов в датафрейме --
  описательные статистики. Наблюдаются ли серьезные отличия в медианных
  и средних значениях показателей (один из признаков наличия аномальных
  наблюдений или выбросов)? Насколько велик разброс значений цен на
  квартиры, если мы сравним минимальное значение и максимальное? Есть ли
  в каком-нибудь показателе с содержательной точки зрения нетипичные
  значения (отрицательные цены и прочее)?
\end{enumerate}

\begin{Shaded}
\begin{Highlighting}[]
\FunctionTok{summary}\NormalTok{(dat)}
\end{Highlighting}
\end{Shaded}

\begin{verbatim}
##       OBS.           TOWN               TOWN.           TRACT     
##  Min.   :  1.0   Length:506         Min.   : 0.00   Min.   :   1  
##  1st Qu.:127.2   Class :character   1st Qu.:26.25   1st Qu.:1303  
##  Median :253.5   Mode  :character   Median :42.00   Median :3394  
##  Mean   :253.5                      Mean   :47.53   Mean   :2700  
##  3rd Qu.:379.8                      3rd Qu.:78.00   3rd Qu.:3740  
##  Max.   :506.0                      Max.   :91.00   Max.   :5082  
##       LON              LAT             MEDV           CMEDV      
##  Min.   :-71.29   Min.   :42.03   Min.   : 5.00   Min.   : 5.00  
##  1st Qu.:-71.09   1st Qu.:42.18   1st Qu.:17.02   1st Qu.:17.02  
##  Median :-71.05   Median :42.22   Median :21.20   Median :21.20  
##  Mean   :-71.06   Mean   :42.22   Mean   :22.53   Mean   :22.53  
##  3rd Qu.:-71.02   3rd Qu.:42.25   3rd Qu.:25.00   3rd Qu.:25.00  
##  Max.   :-70.81   Max.   :42.38   Max.   :50.00   Max.   :50.00  
##       CRIM                ZN             INDUS            CHAS        
##  Min.   : 0.00632   Min.   :  0.00   Min.   : 0.46   Min.   :0.00000  
##  1st Qu.: 0.08205   1st Qu.:  0.00   1st Qu.: 5.19   1st Qu.:0.00000  
##  Median : 0.25651   Median :  0.00   Median : 9.69   Median :0.00000  
##  Mean   : 3.61352   Mean   : 11.36   Mean   :11.14   Mean   :0.06917  
##  3rd Qu.: 3.67708   3rd Qu.: 12.50   3rd Qu.:18.10   3rd Qu.:0.00000  
##  Max.   :88.97620   Max.   :100.00   Max.   :27.74   Max.   :1.00000  
##       NOX               RM             AGE              DIS        
##  Min.   :0.3850   Min.   :3.561   Min.   :  2.90   Min.   : 1.130  
##  1st Qu.:0.4490   1st Qu.:5.886   1st Qu.: 45.02   1st Qu.: 2.100  
##  Median :0.5380   Median :6.208   Median : 77.50   Median : 3.207  
##  Mean   :0.5547   Mean   :6.285   Mean   : 68.57   Mean   : 3.795  
##  3rd Qu.:0.6240   3rd Qu.:6.623   3rd Qu.: 94.08   3rd Qu.: 5.188  
##  Max.   :0.8710   Max.   :8.780   Max.   :100.00   Max.   :12.127  
##       RAD              TAX           PTRATIO            B         
##  Min.   : 1.000   Min.   :187.0   Min.   :12.60   Min.   :  0.32  
##  1st Qu.: 4.000   1st Qu.:279.0   1st Qu.:17.40   1st Qu.:375.38  
##  Median : 5.000   Median :330.0   Median :19.05   Median :391.44  
##  Mean   : 9.549   Mean   :408.2   Mean   :18.46   Mean   :356.67  
##  3rd Qu.:24.000   3rd Qu.:666.0   3rd Qu.:20.20   3rd Qu.:396.23  
##  Max.   :24.000   Max.   :711.0   Max.   :22.00   Max.   :396.90  
##      LSTAT      
##  Min.   : 1.73  
##  1st Qu.: 6.95  
##  Median :11.36  
##  Mean   :12.65  
##  3rd Qu.:16.95  
##  Max.   :37.97
\end{verbatim}

*Столбцы chas, crim, zn имеют серьезные различия между средним и
медианным значениями. Столбец chas означает явно какой-то редкий признак
(среднее крайне близко к 0).Столбцы crim и zn в этом смысле выглядят
более ровныеми (3 квартиль почти равен среднему числу), хотя также
остаются явно скошенными. Crim обладает ярко выбивающимся максимумом по
сравнению с другими аналитиками 4. Используя стандартные средства R (без
\texttt{tidyverse}), добавьте в датафрейм столбец \texttt{MEDV\_N},
который содержит медианные цены на квартиры из столбца \texttt{MEDV},
измеренные в долларах (не в тысячах долларов).

\begin{Shaded}
\begin{Highlighting}[]
\NormalTok{dat}\SpecialCharTok{$}\NormalTok{MEDV\_N }\OtherTok{\textless{}{-}}\NormalTok{ dat}\SpecialCharTok{$}\NormalTok{MEDV }\SpecialCharTok{*} \DecValTok{1000}
\FunctionTok{head}\NormalTok{(dat)}
\end{Highlighting}
\end{Shaded}

\begin{verbatim}
##   OBS.       TOWN TOWN. TRACT      LON     LAT MEDV CMEDV    CRIM ZN INDUS CHAS
## 1    1     Nahant     0  2011 -70.9550 42.2550 24.0  24.0 0.00632 18  2.31    0
## 2    2 Swampscott     1  2021 -70.9500 42.2875 21.6  21.6 0.02731  0  7.07    0
## 3    3 Swampscott     1  2022 -70.9360 42.2830 34.7  34.7 0.02729  0  7.07    0
## 4    4 Marblehead     2  2031 -70.9280 42.2930 33.4  33.4 0.03237  0  2.18    0
## 5    5 Marblehead     2  2032 -70.9220 42.2980 36.2  36.2 0.06905  0  2.18    0
## 6    6 Marblehead     2  2033 -70.9165 42.3040 28.7  28.7 0.02985  0  2.18    0
##     NOX    RM  AGE    DIS RAD TAX PTRATIO      B LSTAT MEDV_N
## 1 0.538 6.575 65.2 4.0900   1 296    15.3 396.90  4.98  24000
## 2 0.469 6.421 78.9 4.9671   2 242    17.8 396.90  9.14  21600
## 3 0.469 7.185 61.1 4.9671   2 242    17.8 392.83  4.03  34700
## 4 0.458 6.998 45.8 6.0622   3 222    18.7 394.63  2.94  33400
## 5 0.458 7.147 54.2 6.0622   3 222    18.7 396.90  5.33  36200
## 6 0.458 6.430 58.7 6.0622   3 222    18.7 394.12  5.21  28700
\end{verbatim}

\begin{enumerate}
\def\labelenumi{\arabic{enumi}.}
\setcounter{enumi}{4}
\item
  Используя стандартные средства R (без \texttt{tidyverse}), сохраните в
  датафрейм \texttt{small} только те строки, которые соответствуют
  пригородам, где медианная цена за квартиру (\texttt{MEDV}) больше 400
  тысяч, но меньше 500 тысяч. Сколько таких пригородов?

  \emph{Подсказка:} для определения числа строк в датафрейме можно
  воспользоваться функцией \texttt{nrow()}.
\end{enumerate}

\begin{Shaded}
\begin{Highlighting}[]
\NormalTok{small }\OtherTok{\textless{}{-}}\NormalTok{ dat[(dat}\SpecialCharTok{$}\NormalTok{MEDV}\SpecialCharTok{\textless{}}\DecValTok{500} \SpecialCharTok{\&}\NormalTok{ dat}\SpecialCharTok{$}\NormalTok{MEDV}\SpecialCharTok{\textgreater{}}\DecValTok{400}\NormalTok{),]}
\FunctionTok{head}\NormalTok{(small)}
\end{Highlighting}
\end{Shaded}

\begin{verbatim}
##  [1] OBS.    TOWN    TOWN.   TRACT   LON     LAT     MEDV    CMEDV   CRIM   
## [10] ZN      INDUS   CHAS    NOX     RM      AGE     DIS     RAD     TAX    
## [19] PTRATIO B       LSTAT   MEDV_N 
## <0 rows> (or 0-length row.names)
\end{verbatim}

\begin{Shaded}
\begin{Highlighting}[]
\FunctionTok{print}\NormalTok{(}\FunctionTok{nrow}\NormalTok{(small))}
\end{Highlighting}
\end{Shaded}

\begin{verbatim}
## [1] 0
\end{verbatim}

\begin{itemize}
\tightlist
\item
  нет таких пригородов
\end{itemize}

\begin{enumerate}
\def\labelenumi{\arabic{enumi}.}
\setcounter{enumi}{5}
\tightlist
\item
  Используя средства библиотеки \texttt{tidyverse} (\texttt{dplyr}),
  создайте столбец \texttt{MEDV\_LOG}, который содержит натуральные
  логарифмы значений медианных цен на квартиры из столбца \texttt{MEDV}.
\end{enumerate}

\begin{Shaded}
\begin{Highlighting}[]
\FunctionTok{library}\NormalTok{(tidyverse)}
\end{Highlighting}
\end{Shaded}

\begin{verbatim}
## -- Attaching packages --------------------------------------- tidyverse 1.3.0 --
\end{verbatim}

\begin{verbatim}
## v ggplot2 3.3.3     v purrr   0.3.4
## v tibble  3.0.6     v dplyr   1.0.4
## v tidyr   1.1.2     v stringr 1.4.0
## v readr   1.4.0     v forcats 0.5.1
\end{verbatim}

\begin{verbatim}
## -- Conflicts ------------------------------------------ tidyverse_conflicts() --
## x dplyr::filter() masks stats::filter()
## x dplyr::lag()    masks stats::lag()
\end{verbatim}

\begin{Shaded}
\begin{Highlighting}[]
\NormalTok{dat }\OtherTok{\textless{}{-}}\NormalTok{ dat }\SpecialCharTok{\%\textgreater{}\%} \FunctionTok{mutate}\NormalTok{(}\AttributeTok{MEDV\_LOG =} \FunctionTok{log}\NormalTok{(MEDV))}
\FunctionTok{head}\NormalTok{(dat)}
\end{Highlighting}
\end{Shaded}

\begin{verbatim}
##   OBS.       TOWN TOWN. TRACT      LON     LAT MEDV CMEDV    CRIM ZN INDUS CHAS
## 1    1     Nahant     0  2011 -70.9550 42.2550 24.0  24.0 0.00632 18  2.31    0
## 2    2 Swampscott     1  2021 -70.9500 42.2875 21.6  21.6 0.02731  0  7.07    0
## 3    3 Swampscott     1  2022 -70.9360 42.2830 34.7  34.7 0.02729  0  7.07    0
## 4    4 Marblehead     2  2031 -70.9280 42.2930 33.4  33.4 0.03237  0  2.18    0
## 5    5 Marblehead     2  2032 -70.9220 42.2980 36.2  36.2 0.06905  0  2.18    0
## 6    6 Marblehead     2  2033 -70.9165 42.3040 28.7  28.7 0.02985  0  2.18    0
##     NOX    RM  AGE    DIS RAD TAX PTRATIO      B LSTAT MEDV_N MEDV_LOG
## 1 0.538 6.575 65.2 4.0900   1 296    15.3 396.90  4.98  24000 3.178054
## 2 0.469 6.421 78.9 4.9671   2 242    17.8 396.90  9.14  21600 3.072693
## 3 0.469 7.185 61.1 4.9671   2 242    17.8 392.83  4.03  34700 3.546740
## 4 0.458 6.998 45.8 6.0622   3 222    18.7 394.63  2.94  33400 3.508556
## 5 0.458 7.147 54.2 6.0622   3 222    18.7 396.90  5.33  36200 3.589059
## 6 0.458 6.430 58.7 6.0622   3 222    18.7 394.12  5.21  28700 3.356897
\end{verbatim}

\begin{enumerate}
\def\labelenumi{\arabic{enumi}.}
\setcounter{enumi}{6}
\tightlist
\item
  Используя средства библиотеки \texttt{tidyverse} (\texttt{dplyr}),
  выведите на экран в отдельной вкладке строки, которые соответствуют
  пригородам, где процент населения, низкого по материальному статусу,
  составляет не менее 30\%.
\end{enumerate}

\begin{Shaded}
\begin{Highlighting}[]
\NormalTok{dat }\SpecialCharTok{\%\textgreater{}\%} \FunctionTok{filter}\NormalTok{(LSTAT }\SpecialCharTok{\textgreater{}=} \DecValTok{30}\NormalTok{) }\SpecialCharTok{\%\textgreater{}\%}\NormalTok{ View}
\end{Highlighting}
\end{Shaded}

\begin{enumerate}
\def\labelenumi{\arabic{enumi}.}
\setcounter{enumi}{7}
\tightlist
\item
  Используя стандартные средства R (без \texttt{ggplot2}), постройте
  гистограмму для показателя \texttt{LSTAT}. Добавьте заголовок и
  подписи к осям, измените цвет. Что можно сказать о распределении
  процента населения с низким материальным статусом? Какие значения
  преобладают, есть ли скошенность в сторону слишком больших или слишком
  маленьких значений? Сохраните полученный график в файл
  \texttt{hist.png}, используя код R (не кнопку \emph{Export} в окне с
  графиком).
\end{enumerate}

\begin{Shaded}
\begin{Highlighting}[]
\FunctionTok{hist}\NormalTok{(dat}\SpecialCharTok{$}\NormalTok{LSTAT,}
     \AttributeTok{main =} \StringTok{"Poor people"}\NormalTok{,}
     \AttributeTok{las =} \DecValTok{2}\NormalTok{,}
     \AttributeTok{breaks =} \DecValTok{20}\NormalTok{,}
     \AttributeTok{xlab =} \StringTok{\textquotesingle{}\% of poor people\textquotesingle{}}\NormalTok{,}
     \AttributeTok{col =} \StringTok{"chocolate"}\NormalTok{)}
\end{Highlighting}
\end{Shaded}

\includegraphics{mid_anl_R3_hw_files/figure-latex/unnamed-chunk-10-1.pdf}

\begin{Shaded}
\begin{Highlighting}[]
\FunctionTok{dev.copy}\NormalTok{(jpeg, }\StringTok{"D:/Skillbox/Analytics, Middle/graph.jpeg"}\NormalTok{)}
\end{Highlighting}
\end{Shaded}

\begin{verbatim}
## jpeg 
##    3
\end{verbatim}

*Достаточно много пригородов, где количество крайне мало, есть
скошенность влево (т.е. в основном пригороды ``богатых'' людей) 9.
Выполните предыдущий пункт, но уже используя средства библиотеки
\texttt{tidyverse} (графика с \texttt{ggplot2}).

\begin{Shaded}
\begin{Highlighting}[]
\FunctionTok{library}\NormalTok{(ggplot2)}
\FunctionTok{ggplot}\NormalTok{(}\AttributeTok{data =}\NormalTok{ dat, }\FunctionTok{aes}\NormalTok{(}\AttributeTok{x =}\NormalTok{ LSTAT)) }\SpecialCharTok{+} 
  \FunctionTok{geom\_histogram}\NormalTok{(}\AttributeTok{binwidth=}\DecValTok{2}\NormalTok{, }\AttributeTok{fill =} \StringTok{"red"}\NormalTok{, }\AttributeTok{color =} \StringTok{"navy"}\NormalTok{)}
\end{Highlighting}
\end{Shaded}

\includegraphics{mid_anl_R3_hw_files/figure-latex/unnamed-chunk-11-1.pdf}

\begin{Shaded}
\begin{Highlighting}[]
\FunctionTok{dev.copy}\NormalTok{(jpeg, }\StringTok{"D:/Skillbox/Analytics, Middle/graph\_gg.jpeg"}\NormalTok{)}
\end{Highlighting}
\end{Shaded}

\begin{verbatim}
## jpeg 
##    4
\end{verbatim}

\hypertarget{ux437ux430ux434ux430ux43dux438ux435-2}{%
\subsubsection{Задание 2}\label{ux437ux430ux434ux430ux43dux438ux435-2}}

\begin{enumerate}
\def\labelenumi{\arabic{enumi}.}
\tightlist
\item
  Загрузите данные из файла \texttt{flats.csv} и сохраните в датафрейм.
  Посмотрите на датафрейм.
\end{enumerate}

\begin{Shaded}
\begin{Highlighting}[]
\NormalTok{an\_dat }\OtherTok{\textless{}{-}} \FunctionTok{read.csv}\NormalTok{(}\StringTok{\textquotesingle{}D:/Skillbox/Analytics, Middle/flats.csv\textquotesingle{}}\NormalTok{)}
\end{Highlighting}
\end{Shaded}

\begin{enumerate}
\def\labelenumi{\arabic{enumi}.}
\setcounter{enumi}{1}
\tightlist
\item
  Сгруппируйте данные по показателю \texttt{brick} (дом из кирпича или
  нет). Определите, сколько домов каждого типа присутствует в данных.
  Каких домов больше?
\end{enumerate}

\begin{Shaded}
\begin{Highlighting}[]
\NormalTok{an\_dat }\SpecialCharTok{\%\textgreater{}\%} \FunctionTok{group\_by}\NormalTok{(brick) }\SpecialCharTok{\%\textgreater{}\%} \FunctionTok{summarise}\NormalTok{(}\AttributeTok{count =} \FunctionTok{n}\NormalTok{())}
\end{Highlighting}
\end{Shaded}

\begin{verbatim}
## # A tibble: 2 x 2
##   brick count
## * <int> <int>
## 1     0  1381
## 2     1   659
\end{verbatim}

\begin{itemize}
\tightlist
\item
  больше не кирпичных домов
\end{itemize}

\begin{enumerate}
\def\labelenumi{\arabic{enumi}.}
\setcounter{enumi}{2}
\tightlist
\item
  Сгруппируйте данные по показателю \texttt{brick} (дом из кирпича или
  нет). Определите среднюю цену на квартиры (\texttt{price}) по каждой
  группе. Квартиры в каких домах, в среднем, дороже?
\end{enumerate}

\begin{Shaded}
\begin{Highlighting}[]
\NormalTok{an\_dat }\SpecialCharTok{\%\textgreater{}\%} \FunctionTok{group\_by}\NormalTok{(brick) }\SpecialCharTok{\%\textgreater{}\%} \FunctionTok{summarise}\NormalTok{(}\AttributeTok{avr\_price =} \FunctionTok{mean}\NormalTok{(price))}
\end{Highlighting}
\end{Shaded}

\begin{verbatim}
## # A tibble: 2 x 2
##   brick avr_price
## * <int>     <dbl>
## 1     0      118.
## 2     1      147.
\end{verbatim}

\begin{itemize}
\tightlist
\item
  Дороже квартиры из кирпича
\end{itemize}

\begin{enumerate}
\def\labelenumi{\arabic{enumi}.}
\setcounter{enumi}{3}
\tightlist
\item
  С помощью библиотеки \texttt{ggplot2} постройте гистограммы для цен на
  квартиры с разбиением на группы по показателю \texttt{walk} (находится
  ли дом в шаговой доступности от метро или нет). По группам --
  отдельное окно-фасетка для каждой группы в рамках одного графика.
  Распределение цен в какой группе менее скошенное (вправо или влево)?
\end{enumerate}

\begin{Shaded}
\begin{Highlighting}[]
\FunctionTok{ggplot}\NormalTok{(}\AttributeTok{data =}\NormalTok{ an\_dat, }\FunctionTok{aes}\NormalTok{(}\AttributeTok{x =}\NormalTok{ price)) }\SpecialCharTok{+}
\FunctionTok{geom\_histogram}\NormalTok{(}\AttributeTok{binwidth =} \DecValTok{10}\NormalTok{, }\AttributeTok{fill =} \StringTok{"yellow"}\NormalTok{, }\AttributeTok{color =} \StringTok{"navy"}\NormalTok{) }\SpecialCharTok{+}
\FunctionTok{facet\_wrap}\NormalTok{(}\SpecialCharTok{\textasciitilde{}}\NormalTok{walk)}
\end{Highlighting}
\end{Shaded}

\includegraphics{mid_anl_R3_hw_files/figure-latex/unnamed-chunk-15-1.pdf}
* С учетом разницы в количестве (больше домов у метро) у обоих графиков
примерно одинаковая скошенность в левую сторону 5. С помощью библиотеки
\texttt{ggplot2} постройте диаграммы рассеяния для показателей
\texttt{totsp} (общая площадь квартиры в квадратных метрах) и
\texttt{price} (цена квартиры), сделав цвет точек зависимым от
показателя \texttt{walk}, а размер -- от показателя \texttt{kitsp}
(площадь кухни).

\begin{Shaded}
\begin{Highlighting}[]
\FunctionTok{ggplot}\NormalTok{(}\AttributeTok{data=}\NormalTok{an\_dat, }\FunctionTok{aes}\NormalTok{(}\AttributeTok{x =}\NormalTok{ totsp, }\AttributeTok{y =}\NormalTok{ price, }\AttributeTok{color =}\NormalTok{ walk, }\AttributeTok{size =}\NormalTok{ kitsp)) }\SpecialCharTok{+} \FunctionTok{geom\_point}\NormalTok{()}
\end{Highlighting}
\end{Shaded}

\includegraphics{mid_anl_R3_hw_files/figure-latex/unnamed-chunk-16-1.pdf}

\begin{verbatim}
*Подсказка:* с размером точек можно поступать точно так же, как с цветом, либо указывать внутри `geom_point()`, либо внутри `aes()` в зависисимости от постановки задачи. Аргумент для настройки размера точки – `size`.
\end{verbatim}

\end{document}
